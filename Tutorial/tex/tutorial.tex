\documentclass[a4paper,DIV12,fleqn]{scrartcl}
\usepackage{amsmath}
\usepackage{amssymb}
\usepackage{mathtools}
\usepackage[utf8]{inputenc}
\usepackage{physics}
\usepackage{hyperref}


\begin{document}

\begin{center}
  {\Huge HEP Computing Tools - Tutorial}\\\vspace{1cm}
  Dr. Rene Poncelet, Cavendish Laboratory Cambridge, June 2022
\end{center}


\paragraph{Introduction:}
The aim of this tutorial is to provide insights in the usage of
\textsc{Sherpa}, a popular Multi-Purpose Monte-Carlo Event Generator
(abbreviated as MC), and a tool chain facilitating the analysis of
MC-events. It is not meant to be a complete and general overview over the
features, caveats and applications of \textsc{Sherpa} or HEP tools in general,
but rather a deep-dive tour which tries to touch as many aspects of
MC event generation and analysis as possible and to give them context.
The tutorial covers the following topics:

\begin{itemize}
  \item Event generation with \textsc{Sherpa}:
    \begin{itemize}
      \item Fixed-order Events (LO,NLO)
      \item Parton-Shower and Hadronization
    \end{itemize}
  \item Event analysis with the help of \textsc{Rivet}:
    \begin{itemize}
      \item Event selection
      \item Histograms
      \item Plotting
    \end{itemize}
\end{itemize}

For this propose we consider the production of top-quark pairs at the
Large-Hadron-Collider, a Standard-Model process which is studied in
great detail by researchers around the world.

\paragraph{Note:} This exercise sheet is part of the HEP Computing Tools
tutorial and works together with the provided virtual machine
\emph{HEPComputingTutorial}. It assumes basic knowledge of working in the
unix bash and with \verb|c++| apart from that it is supposed to be
self explanatory. The tutorial uses the following software packages:
\begin{itemize}
  \item \textsc{Sherpa}:\\
        documentation:
          \url{https://sherpa.hepforge.org/doc/SHERPA-MC-2.2.7.html}\\
        source code: \url{https://sherpa-team.gitlab.io/})\\
        release publication: \cite{Gleisberg:2008ta}
  \item Rivet (\url{https://rivet.hepforge.org/}) \cite{Bierlich:2019rhm},
        and plugins
  \item LHAPDF (\url{https://lhapdf.hepforge.org/}) \cite{Buckley:2014ana}
  \item OpenLoops 2 (\url{https://openloops.hepforge.org/})
        \cite{Buccioni:2019sur}
\end{itemize}
Usually the first step in dealing with MC generators and related tools
is to install the software. This is usually a tedious exercise due to
the various dependencies.
The virtual machine provides a environment where all tools and software
packages for this tutorial have been install already, and therefore
it is strongly
recommended to use this virtual machine for the tutorial. If you are
interested in installing the same tool chain on your private system, please
referee to the shell-script \verb|~/HEPSoftware/install-all.sh|.


One additional information: the username within the virtual machine
is \emph{tutorial} and the password is \emph{1234test}. This is in case
you need sudo rights.

\section{Exercise 1 - Top-quark pair production}
\label{sec:task1}

Please go to the directory \verb|~/Tutorial/exercise1|.

The \textsc{Sherpa} software is configured by a so called \emph{run-card}.
In this case it is called \verb|Run.dat| and can be found in the directory.
A run card consists of the following sections:
\begin{itemize}
  \item \verb|(run)|:
               In this section parameter and general configurations appear.
               Relevant for
               any calculations are the number of events to be generated
               (\verb|EVENTS|) and the matrix element generator
               (\verb|ME_SIGNAL_GENERATOR|) which is chosen to be the default here.
               Also the type and energy of the colliding particles has to be
               specified (\verb|BEAM 2122| means proton beam, \verb|BEAM_ENERGY|
               its laboratory frame energy). A further important parameter to be
               set is the renormalization and factorization scale (\verb|SCALE|)
               which is set to be the top-quark mass.
  \item \verb|(processes)|:
               In this section the signal processes are specified. Since
               we are interested of the production of a top-quark pair
               the signal processes is \verb|93 93 -> 6 -6| where 93 stands for
               any massless quark or gluon and 6 the top-quark (PDG-ID). The
               perturbative order in $\alpha_s$ and $\alpha_{EW}$
               can also be specified, \verb|Order(*,0)| just means the lowest order
               in $\alpha_s$ assuming $\alpha_{EW}^0$.
  \item \verb|(analysis)|: In this section the analyses are specified. In this example
               we specify to use the \textsc{Rivet} back-end. The analysis name
               \verb|TUTORIAL_TTBAR_0| selects a analysis designed for this
               exercise.
\end{itemize}
In general the \textsc{Sherpa} documentation gives information about all kinds of
parameters and steering options.

\textsc{Sherpa} is started by the command
\begin{verbatim}
  Sherpa -f Run.dat
\end{verbatim}
The first time it will stop with the comment that you have to make the
matrix element libraries (\verb|./makelibs|). Afterwards \textsc{Sherpa}
should start and should go through various steps of preparation until
the event generation eventually starts. The analysis \verb|TUTORIAL_TTBAR_0|
just prints out the final state particles (their PID) and their momenta, you
should see them printed to the prompt. How does the particle content change if:
\begin{enumerate}
  \item You turn on the top-quark decays (see corresponding commented lines
        in the run-card)?
  \item You turn on the parton-shower (see corresponding commented lines in
        the run-card)?
  \item You turn on the hadronization (see corresponding commented lines in
        the run-card)? (This might take a minute or two due to additional
        preparation steps.)
\end{enumerate}

Looking at the final state particles and their momenta is nice but is not very
illuminating. We want to write now our own analysis so we can study the events
in a statistical way. For this, go to the directory \verb|~/Tutorial/analyses|
and find the \verb|TUTORIAL_TTBAR_1.cc| file. This file contains the implementation
of a simple analysis of the top-quark/anti-top-quark transverse momentum
based on partonic top-quarks.
\begin{enumerate}
  \item Add some new observables (like the invariant mass of the top-quark
        pair, rapidities) to the analysis.
        Hint: if you want to know how to access different observables from
        the \verb|particle| class, type:
        \begin{verbatim}
          firefox ~/manual/Rivet/index.html
        \end{verbatim}
  \item Compile it using the \verb|compile.sh| script in the same directory
  \item Go back to the exercise 1 directory and restore the \verb|Run.dat|
        (there is a \verb|Run.dat-save|) original state and replace the
        \verb|TUTORIAL_TTBAR_0| analysis by the \verb|TUTORIAL_TTBAR_1|
        analysis.
        We will need also more events, 100k to 1M will be enough. Run Sherpa.
  \item If everything worked out there will be a \verb|Analysis.yoda| file.
        You can create plots using the \textsc{Rivet} tool \verb|rivet-mkhtml|.
        Invoking \verb|rivet-mkhtml Analysis.yoda| will create a webpage where
        you can have a look at your plots.
  \item Try to see how the results change when activating the parton-shower.
        Keep in mind to save your \verb|Analysis.yoda| of the previous run.
        You can directly compare your two results with \verb|rivet-mkhtml|
        (see \verb|rivet-mkhtml --help|).
\end{enumerate}

\section{Exercise 2 - Top-quark mass measurement}
\label{sec:task2}

The aim of this exercise is perform a phenomenological analysis
of top-quark pair events with an emphasis on the reconstruction
of the top-quarks from decay products.


Please go to the directory \verb|~/Tutorial/exercise2|.


To start you find again a \verb|Run.dat|. It is roughly the same as in
exercise 1 but it includes now a 'professional' top-quark
pair analysis \verb|MC_TUTORIAL_TTBAR|. First try to run the setup and
investigate the \verb|Analysis.yoda|. This analysis contains many observables
which are of interest for a top-quark analysis. In particular it contains
a top-quark mass measurement (\verb|onelep_t_mass|).
\begin{enumerate}
  \item How does the result change if you deactivate hadronization?
  \item How does the result change if you deactivate the parton shower?
\end{enumerate}


The next step is to perform such a 'measurement' yourself. Go to
the \verb|~/Tutorial/analyses/| directory. A \verb|TUTORIAL_TTBAR_2|
analysis is prepared for you to be modified. There is also
a \verb|MC_TUTORIAL_TTBAR| which is the implementation of the
analysis we studied in the first part of this exercise.
\begin{enumerate}
  \item Modify \verb|TUTORIAL_TTBAR_2| to perform a top-quark mass
        measurement. If you need some ideas or inspirations have look at
        the \verb|MC_TUTORIAL_TTBAR|.
  \item Compare your results with those from \verb|MC_TUTORIAL_TTBAR|.
\end{enumerate}

\section{Exercise 3 - Homework, Top-quark pair production beyond Born
         approximation}
\label{sec:task3}
Please go to the directory \verb|~/Tutorial/exercise3|.

The goal of this exercise is to perform a next-to-leading order (NLO)
calculation including a matching to a parton shower.
The preparation for such a calculations
is usually much more involved than a simple leading order computation
and thus cannot be done fully in the tutorial itself. If you are
interested in performing it yourself I recommend to do it as a homework.

The \verb|Run.dat| file has been modified to incorporate the NLO 
corrections to top-quark pair production including the matching to
the parton-shower in the MC@NLO scheme. Running this setup for the first
time will $\sim$ 3 hours. You can use the same \verb|TUTORIAL_TTBAR_1|
analysis to study the events.

\begin{enumerate}
  \item Compare the results obtained at NLO with the results obtained
        in the first exercise. What do you observe?
  \item You can extend the \textsc{Sherpa} setup such that is simulates
        top-quark decays. You can run the analysis from exercise
        2 and compare the results.
\end{enumerate}

\begin{thebibliography}{99}
%\cite{Bierlich:2019rhm}
\bibitem{Bierlich:2019rhm} 
  C.~Bierlich \textit{ et al.},
  %``Robust Independent Validation of Experiment and Theory: Rivet version 3,''
  SciPost Phys.\  \textbf{8}, 026 (2020)
  doi:10.21468/SciPostPhys.8.2.026
  [arXiv:1912.05451 [hep-ph]].
  %%CITATION = doi:10.21468/SciPostPhys.8.2.026;%%
  %3 citations counted in INSPIRE as of 05 Mar 2020

%\cite{Gleisberg:2008ta}
\bibitem{Gleisberg:2008ta} 
  T.~Gleisberg, S.~Hoeche, F.~Krauss, M.~Schonherr, S.~Schumann, F.~Siegert and J.~Winter,
  %``Event generation with SHERPA 1.1,''
  JHEP \textbf{0902}, 007 (2009)
  doi:10.1088/1126-6708/2009/02/007
  [arXiv:0811.4622 [hep-ph]].
  %%CITATION = doi:10.1088/1126-6708/2009/02/007;%%
  %2782 citations counted in INSPIRE as of 05 Mar 2020

%\cite{Buckley:2014ana}
\bibitem{Buckley:2014ana} 
  A.~Buckley, J.~Ferrando, S.~Lloyd, K.~Nordström, B.~Page, M.~Rüfenacht, M.~Schönherr and G.~Watt,
  %``LHAPDF6: parton density access in the LHC precision era,''
  Eur.\ Phys.\ J.\ C \textbf{ 75}, 132 (2015)
  doi:10.1140/epjc/s10052-015-3318-8
  [arXiv:1412.7420 [hep-ph]].
  %%CITATION = doi:10.1140/epjc/s10052-015-3318-8;%%
  %673 citations counted in INSPIRE as of 05 Mar 2020

%\cite{Buccioni:2019sur}
\bibitem{Buccioni:2019sur} 
  F.~Buccioni, J.~N.~Lang, J.~M.~Lindert, P.~Maierhöfer, S.~Pozzorini, H.~Zhang and M.~F.~Zoller,
  %``OpenLoops 2,''
  Eur.\ Phys.\ J.\ C \textbf{79}, no. 10, 866 (2019)
  doi:10.1140/epjc/s10052-019-7306-2
  [arXiv:1907.13071 [hep-ph]].
  %%CITATION = doi:10.1140/epjc/s10052-019-7306-2;%%
  %16 citations counted in INSPIRE as of 05 Mar 2020
\end{thebibliography}
\end{document}
